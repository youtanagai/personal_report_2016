% 以下のコメントはよく読んだほうが良いと思います。
\documentclass[twocolumn]{jsarticle}

\usepackage{graphicx,color}%図版を取り込む場合は必要。
\usepackage{okumacro}% 振り仮名を使用するときは必要。
%
\ifdraft
  \def\hissu{\bgroup\color{red}}
  \def\endhissu{\egroup}
\else
  \def\hissu{}
  \def\endhissu{}
\fi
% % % % % % % % % % % % % % % % % % % % % % % % % % % % % %
% ここから↓
\pagestyle{empty}
\advance\textheight\headheight \headheight=0pt
\advance\textheight\headsep    \headsep=0pt
\advance\textheight\footskip   \footskip=0pt
\textheight=738truept
\advance\textwidth\marginparsep \marginparsep=0pt
\advance\textwidth\marginparwidth \marginparwidth=0pt
\advance\textwidth\oddsidemargin \oddsidemargin=0pt
\evensidemargin=\oddsidemargin
\textwidth=50zw
\advance\textwidth2zw
\columnsep=2zw
\topmargin=-5.4mm
\oddsidemargin=-7.4mm
\begin{document}
\fontsize{10}{18}\selectfont
% ↑ここまで変更しないほうが良いです。
% % % % % % % % % % % % % % % % % % % % % % % % % % % % % %

% ↓ここからが変更すべき箇所です。
\twocolumn[%
% 中間か最終かを記載する
\noindent{個人報告書(中間)Personal Middle Report}
% 個々に提出日を記入する
   \hfill{提出日 (Date) 20016/07/26}\par\vskip2zw
\begin{center}
%日本語のプロジェクト名
 {\LARGE 使ってもらって学ぶフィールド指向システム・デザイン}\par\vskip1zw
%英語のプロジェクト名
 {\LARGE Field Oriented System Design Learning by Users' Feedback}\par\vskip1.5zw
%グループ名
 {\large 町内会グループ (A)\quad Neighborhood Association Group (A)}\par\vskip1zw
%学籍番号、日英の名前
 {\large b1014120\quad 永井陽太\quad Yota Nagai}\par\vskip2zw
\end{center}]%

% \begin{hissu} ... \end{hissu} で囲まれている部分、
% 必須項目でありますが、そのまま記述してはいけません。
% 最終的には削除してもかまいませんが、コメントアウトして、
% 原稿に残しておくと良いでしょう。
% たとえば次のように。
%\begin{hissu}
%ここでグループが取組む課題の背景を簡略に記述する。
%以下の項目を含む内容にすることが望ましい。
%\begin{itemize}
% \item 当該分野の現状や従来例 。
% \item 現状における問題点。
%\end{itemize}
%\end{hissu}

%背景
\section{背景}
函館市陣川町にある陣川あさひ町会の現状

 陣川あさひ町会は参加者が1000人にもなるイベントを開催するなど積極的に活動している町会である。
陣川あさひ町会が抱える問題

 イベント情報の投稿に2つのツール(FacebookとLINE@)を利用しているので手間がかかる。
 町民のイベントへの参加申し込み方法が3つあるので、町会側での参加者管理作業に時間がかかる。
 イベントに関する緊急連絡(例えば悪天候による中止の連絡等)が迅速に行えていない。


%課題の設定と到達目標
\input purpose.tex
%\begin{hissu}
%個人の課題・到達目標がグループ内でどのような位置にあ
%るのかを明確にする。以下の順で説明するとわかりやすい。
%\begin{itemize}
% \item グループが取組む課題の設定と到達目標。
% \item 自分自身が取組む課題の設定と到達目標。
%\end{itemize}
%\end{hissu}

%課題解決のプロセスとその結果
\input process.tex
%\begin{hissu}
%個人の課題を解決するためにどのような作業を行ったか? 創意工夫
%はあるか? などを記述する。さらにそれらの作業で得られた結果・成
%果をわかりやすく説明する。図や表の利用を薦める。
%\end{hissu}
%\begin{table}[htbp]
%\begin{center}
%\caption{個人報告書の文字サイズ・文字数・行数}\label{tab:hoge}
%\begin{tabular}{ccc}
%\hline
%文字サイズ & 1行あたりの文字数& 行数\\
%\hline
%9ポイント  & 24--28文字       & 25--32行\\
%\hline
%\end{tabular}
%\end{center}
%\end{table}
%\begin{hissu}
%フォントサイズ、字送り、行送りなどは表~\ref{tab:hoge}の
%文字サイズ・文字数・行数に従うこと
%\end{hissu}

%今後の課題
\section{今後の課題}
我々は、役員によるイベント情報の発信機能、緊急連絡機能、アプリ所有者への通知機能を備えたプロトタイプを作成する。
その後、陣川あさひ町会が8月6日、7日に開催する「納涼まつり」にてプロトタイプのデモとアンケートを行いプロトタイプを評価して頂く。
また、後期ではアプリケーションの開発を進めていきながら、町民がアプリを利用したくなるようなコンテンツの考案と、FacebookとLINE@との連携機能の追加を行っていく。

%\begin{hissu}
% 得られた結果・成果の不十分な点を指摘し、今
% 後どのような作業を行うか記述する。
%\end{hissu}

%リファレンス

\begin{thebibliography}{2}
\bibitem{YF2000}
{\ruby{永井}{}}{\ruby{勝則}{}}.
\newblock クラウドで出来るHTML5ハイブリッドアプリ開発 Cordova/Onsen UIで作るiOS/Android両対応アプリ (Monaca公式ガイドブック) .
\newblock 翔泳社, 2015.

\bibitem{IH2001}
{\ruby{大塚}{}}{\ruby{弘記}{}}.
\newblock github実践入門 Pull Requestによる開発の変革.
\newblock 技術評論社, 2014.

\end{thebibliography}



\end{document}
