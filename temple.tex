% 以下のコメントはよく読んだほうが良いと思います。
\documentclass[twocolumn]{jsarticle}

\usepackage{graphicx,color}%図版を取り込む場合は必要。
\usepackage{okumacro}% 振り仮名を使用するときは必要。
%
\ifdraft
  \def\hissu{\bgroup\color{red}}
  \def\endhissu{\egroup}
\else
  \def\hissu{}
  \def\endhissu{}
\fi
% % % % % % % % % % % % % % % % % % % % % % % % % % % % % %
% ここから↓
\pagestyle{empty}
\advance\textheight\headheight \headheight=0pt
\advance\textheight\headsep    \headsep=0pt
\advance\textheight\footskip   \footskip=0pt
\textheight=738truept
\advance\textwidth\marginparsep \marginparsep=0pt
\advance\textwidth\marginparwidth \marginparwidth=0pt
\advance\textwidth\oddsidemargin \oddsidemargin=0pt
\evensidemargin=\oddsidemargin
\textwidth=50zw
\advance\textwidth2zw
\columnsep=2zw
\topmargin=-5.4mm
\oddsidemargin=-7.4mm
\begin{document}
\fontsize{10}{18}\selectfont
% ↑ここまで変更しないほうが良いです。
% % % % % % % % % % % % % % % % % % % % % % % % % % % % % %

% ↓ここからが変更すべき箇所です。
\twocolumn[%
% 中間か最終かを記載する
\noindent{個人報告書(最終)Personal Final Report}
% 個々に提出日を記入する
   \hfill{提出日 (Date) 2004/??/??}\par\vskip2zw
\begin{center}
%日本語のプロジェクト名
 {\LARGE 函館版おいしいカレーの作り方プロジェクト}\par\vskip1zw
%英語のプロジェクト名
 {\LARGE How to Make Delicious Curry of Hakodate}\par\vskip1.5zw
%グループ名
 {\large 調査グループ (A)\quad Search Group (A)}\par\vskip1zw
%学籍番号、日英の名前
 {\large m780021\quad はこだて次郎\quad Jiro Hakodate}\par\vskip2zw
\end{center}]%

% \begin{hissu} ... \end{hissu} で囲まれている部分、
% 必須項目でありますが、そのまま記述してはいけません。
% 最終的には削除してもかまいませんが、コメントアウトして、
% 原稿に残しておくと良いでしょう。
% たとえば次のように。
%\begin{hissu}
%ここでグループが取組む課題の背景を簡略に記述する。
%以下の項目を含む内容にすることが望ましい。
%\begin{itemize}
% \item 当該分野の現状や従来例 。
% \item 現状における問題点。
%\end{itemize}
%\end{hissu}

\section{背景}
\begin{hissu}
ここでグループが取組む課題の背景を簡略に記述する。
以下の項目を含む内容にすることが望ましい。
\begin{itemize}
 \item 当該分野の現状や従来例 。
 \item 現状における問題点。
\end{itemize}
\end{hissu}

\section{課題の設定と到達目標}
\begin{hissu}
個人の課題・到達目標がグループ内でどのような位置にあ
るのかを明確にする。以下の順で説明するとわかりやすい。
\begin{itemize}
 \item グループが取組む課題の設定と到達目標。
 \item 自分自身が取組む課題の設定と到達目標。
\end{itemize}
\end{hissu}

\section{課題解決のプロセスとその結果}
\begin{hissu}
個人の課題を解決するためにどのような作業を行ったか? 創意工夫
はあるか? などを記述する。さらにそれらの作業で得られた結果・成
果をわかりやすく説明する。図や表の利用を薦める。
\end{hissu}
\begin{table}[htbp]
\begin{center}
\caption{個人報告書の文字サイズ・文字数・行数}\label{tab:hoge}
\begin{tabular}{ccc}
\hline
文字サイズ & 1行あたりの文字数& 行数\\
\hline
9ポイント  & 24--28文字       & 25--32行\\
\hline
\end{tabular}
\end{center}
\end{table}
\begin{hissu}
フォントサイズ、字送り、行送りなどは表~\ref{tab:hoge}の
文字サイズ・文字数・行数に従うこと
\end{hissu}

\section{今後の課題}
\begin{hissu}
 得られた結果・成果の不十分な点を指摘し、今
 後どのような作業を行うか記述する。
\end{hissu}

\begin{hissu}
\begin{thebibliography}{10}
\bibitem{YF2000}
{\ruby{藤沢}{ふじさわ}}{\ruby{幸穂}{ゆきほ}}.
\newblock H8マイコン完全マニュアル.
\newblock オーム社, 2000.

\bibitem{IH2001}
{\ruby{英}{はなぶさ}}{\ruby{一太}{いちお}}.
\newblock プリント配線基板の製造技術.
\newblock CMC, 2001.

\bibitem{TH1999}
{\ruby{堀}{ほり}}{\ruby{敏夫}{としお}}.
\newblock 電源回路の図式解析と設計法.
\newblock 総合電子出版社, 1999.

\bibitem{HH1980}
{\ruby{福与}{ふくよ}}{\ruby{人八}{ひとひろ}},
  {\ruby{小林}{こばやし}}{\ruby{肇}{はじめ}},
  {\ruby{泊川}{とまりがわ}}{\ruby{一之}{かずゆき}}.
\newblock 電子計測 改訂版.
\newblock 実教出版, 1980.

\bibitem{YI2003}
{\ruby{市川}{いちかわ}}{\ruby{裕一}{ゆういち}},
  {\ruby{青木}{あおき}}{\ruby{勝}{まさる}}.
\newblock GHz時代の高周波回路設計.
\newblock CQ出版, 2003.

\bibitem{KI1989}
{\ruby{伊藤}{いとう}}{\ruby{謹司}{きんし}}.
\newblock プリント配線技術読本.
\newblock 日刊工業新聞社, 1989.

\bibitem{TK1995}
{\ruby{加藤}{かとう}}{\ruby{肇}{ただし}},
  {\ruby{見城}{けんじょう}}{\ruby{嵩志}{たかし}},
  {\ruby{高橋}{たかはし}}{\ruby{久}{ひさし}}.
\newblock 図解・わかる電子回路.
\newblock 講談社, 1995.

\bibitem{TK1993}
{\ruby{小西}{こにし}}{\ruby{照郎}{てるお}}.
\newblock 初級プリント回路技術者実力向上講座 (1) プリント回路設計.
\newblock 日刊工業新聞社, 1993.

\bibitem{KK2004}
{\ruby{今野}{こんの}}{\ruby{金顕}{かねあき}}.
\newblock マイコン技術教科書 H8編.
\newblock CQ出版社, 2004.

\bibitem{NK1994}
{\ruby{畔柳}{くろやなぎ}}{\ruby{功芳}{のりよし}},
  {\ruby{塩谷}{しおや}}{\ruby{光}{ひかる}}.
\newblock 通信工学通論.
\newblock コロナ社, 1994.

\bibitem{NM1998}
{\ruby{三上}{みかみ}}{\ruby{直樹}{なおき}}.
\newblock デジタル信号処理の基礎.
\newblock CQ出版社, 1998.

\bibitem{SN1991}
{\ruby{中村}{なかむら}}{\ruby{尚吾}{しょうご}}.
\newblock デジタル信号処理.
\newblock 東京電機大学出版局, 1991.

\bibitem{NO2000}
{\ruby{大澤}{おおさわ}}{\ruby{直}{ただし}}.
\newblock はんだ付けの基礎と応用.
\newblock 工業規格調査会, 2000.

\bibitem{CQ1991}
CQ出版.
\newblock トランジスタ技術SPECIAL No.28.
\newblock CQ出版社, 1991.

\bibitem{IS1991}
{\ruby{相良}{さがら}}{\ruby{岩男}{いわお}}.
\newblock AD/DA変換回路入門.
\newblock 日刊工業新聞社, 1991.

\end{thebibliography}
\end{hissu}

\end{document}
